\documentclass[table]{beamer}

\usepackage{graphicx, graphics}
\usepackage{float, array, color, ctex}
\usepackage{amsmath, amssymb}
\usepackage{multicol, multirow, makecell, tabu, dcolumn}
\usepackage{fancyhdr, lastpage}
\usepackage{listings, xcolor}
\usepackage{xeCJKfntef}
\usepackage{fontspec, xunicode, xltxtra}
\usepackage{setspace}
\usepackage{geometry}

\usepackage[UTF8, noindent]{ctexcap}
\usetheme{Szeged}
\usecolortheme{beaver}
\usefonttheme[math]{serif}

\title{字符串算法}
\author{张敬东}
\institute{福州第十六中学}
\date{\today}

\AtBeginSection[]
{
	\begin{frame}
		\frametitle{目录}
		\tableofcontents[currentsection]
	\end{frame}
}

\begin{document}

\frame{\titlepage} % 生成标题页

\begin{frame} % 生成目录页
	\frametitle{目录}

	\tableofcontents

\end{frame}

\section{一些定义}

\begin{frame}{一些定义}
	\begin{itemize}
		\item 默认字符串下标从 $0$ 开始。
		\pause
		\item 字符串 $s$ 中字符的个数称为\textbf{字符串的长度,记作 $|s|$};
		\pause
		\item $s$ 中第 $i$ 个字符记作 $s_i$。
		\pause
		\item $s$ 的子串 $s[i\dots j]$ 表示字符 $s[i],\dots,s[j]$ 串起来形成的字符串;
		\pause
		\item $s[1\dots i]$ 是 $s$ 到 $i$ 的前缀,$s[i\dots |s|-1]$ 是 $s$ 的后缀。
		
		其中若 $s[1\dots i]\ne s$,则称其为 $s$ 的真前缀。真后缀同理。
		\pause
		\item 回文串是形如 $\texttt{abcba},\texttt{abccba}$ 的字符串。
	\end{itemize}
\end{frame}

\section{KMP}

\begin{frame}{字符串查找}
	有时我们希望在文本串 $t$ 中查找模式串 $s$。比如你按下 Ctrl+F 时浏览器就会在页面中查找你输入的字符串。
	\pause
	
	一种方法是暴力查找,时间复杂度 $O(nm)$。有没有更快的方法呢?
	\pause

	\begin{block}{注意}
		本节中规定:文本串 $t$,模式串 $s$,$m=|t|,n=|s|$。
	\end{block}
\end{frame}

\begin{frame}{前缀函数}
	定义:
	\begin{itemize}
		\item $s$ 的 border:$s$ 的最长公共真前后缀。例如 $\texttt{abccdabc}$ 的 border 为 $\texttt{abc}$。
		\item $s$ 前缀函数 $\pi_i$:$s[1\dots i]$ 的 border 长度。显然 $\pi_0=0$。
	\end{itemize}
	\pause

	对于 $s=\texttt{abcabcd}$,$\pi_0=0,\pi_1=0,\pi_2=0,\pi_3=1,\pi_4=2,\pi_5=3,\pi_6=0$。
	\pause

	\textbf{那么如何计算前缀函数呢?}
\end{frame}

\begin{frame}{前缀函数的计算}
	观察可以发现 $\pi_i$ 至多增加 $1$,也就是说如果考虑现在算 $\pi_i$,那么我们希望最好 $\pi_i=\pi_{i-1}+1$,此时 $s[\pi_{i-1}]=s[i]$。
	\pause
	
	如:
	$$\underbrace{\overbrace{s_0 ~ s_1 ~ s_2}^{\pi_{i-1} = 3} ~ s_3}_{\pi_i = 4} ~ \dots ~ \underbrace{\overbrace{s_{i-3} ~ s_{i-2} ~ s_{i-1}}^{\pi_{i-1}=3} ~ s_{i}}_{\pi_i = 4}$$
	(来自 OI-Wiki。)

	此时 $\pi_{i-1}=3$,我们希望 $s[3]=s[i]$,这样就可以得到 $\pi_i=\pi_{i-1}+1$。
\end{frame}

\begin{frame}
	\textbf{但是},如果 $s[\pi_{i-1}]\ne s[i]$,那怎么办?我们将这种情况称为“失配”。
	\pause

	失配时,我们依然想让 border 尽可能地长,所以我们希望找到一个
	\textbf{严格次长 border}。我们将原 border 记为 $b$,严格次长 border 记为 $b'$。
	\pause

	\begin{enumerate}
		\item 首先 $|b|>|b'|$(所以是“真前后缀”)。
		\pause
		\item 然后又由于 $b'$ 和 $b$ 均是 $s[1\dots i-1]$ 的公共真前后缀,
		故 $b'$ 是 $b$ 的公共真前后缀(可以从下图看出)。
		\textbf{注意,$b$ 和 $b'$ 并不表示长度,而是字符串},左右两对字符串 $b$ 和 $b'$ 是分别相等的。
		左边可以看出 $b'$ 是 $b$ 的前缀,右边可以看出 $b'$ 是 $b$ 的后缀。
		\pause
		\item 因为我们规定是\textbf{严格次长 border},
		所以除了 $b$ 没有其他 border 比 $b'$ 长,又因为 $b'$ 是 $b$ 的公共真前后缀,
		\textbf{所以 $b'$ 是 $b$ 的 border}。
	\end{enumerate}
$$\overbrace{\underbrace{s_0 ~ s_1}_{b'} ~ s_2 ~ s_3}^{b} ~ \dots ~ \overbrace{s_{i-4} ~ s_{i-3} ~ \underbrace{s_{i-2} ~ s_{i-1}}_{b'}}^{b} ~ s_{i}$$
\end{frame}

\begin{frame}{总结:前缀函数}
	所以我们找到了一个递归关系:一个 border 的严格次长 border 为它自己的 border。(好绕啊!)
	前缀函数的求法就出来了:
	\begin{itemize}
		\item 先看是否满足 $s[\pi_{i-1}]=s[i]$。
		\item 如果满足,那么 $\pi_i=\pi_{i-1}+1$。
		\item 否则就是失配,检查严格次长 border 能否匹配:$s[|b'|]=s[i]$。
		\begin{itemize}
			\item 若还是不能,继续上一个操作,使 $b\gets b',b'\gets b~的~\text{border}$。
			(检查严格次长 border 的严格次长 border,或者说检查严格次次长 border)。
			直到无解或匹配上。
			\item 如果匹配,$\pi_{i}=|b'|$。
			\item 如果无解,$\pi_{i}=0$。
		\end{itemize}
	\end{itemize}
\end{frame}

\begin{frame}[fragile]{代码:前缀函数}
	所以最终我们可以构建一个不需要进行任何字符串比较,并且只进行 $O(n)$ 次操作的算法。
	而且该算法的实现出人意料的短且直观:(摘自 OI-Wiki。)
\begin{verbatim}
vector<int> prefix_function(string s) {
    int n = (int)s.length();
    vector<int> pi(n);
    for (int i = 1; i < n; i++) {
        int j = pi[i - 1];
        while (j > 0 && s[i] != s[j]) j = pi[j - 1];
        if (s[i] == s[j]) j++;
        pi[i] = j;
    }
    return pi;
}
\end{verbatim}
\end{frame}

\begin{frame}{KMP}
	说了半天前缀函数,KMP 到底怎么回事?
	\pause

	你先别急,你如果能真的理解前缀函数,那 KMP 就是小菜一碟了。
	\pause

	在文本串 $t$ 中查找模式串 $s$ 时,我们令 $u=s+\#+t$,其中 $+$ 为拼接,
	$\#$ 为任意 $s$ 和 $t$ 中不包含的字符。对 $u$ 求前缀函数,若 $\pi_i=n$,
	那么 $u[i-n+1\dots i]=s$。
	\pause

	细想为什么:若 $\pi_i=n$,那么 $s[1\dots i]$ 的最长公共真前后缀就是 $s$,
	也就是说 $t$ 中也出现了一个完整的 $s$,我们就成功在 $t$ 中查找到 $s$ 啦!
\end{frame}

\begin{frame}[fragile]{代码:KMP}
\begin{verbatim}
    vector<int> find_occurrences(string t, string s) {
        string cur = s + '#' + t;
        int m = t.size(), n = s.size();
        vector<int> v;
        vector<int> lps = prefix_function(cur);
        for (int i = n + 1; i <= m + n; i++) {
            if (lps[i] == n)
                v.push_back(i - 2 * n);
        }
        return v;
    }
\end{verbatim}
\end{frame}

\section{exKMP/Z 函数}

\begin{frame}
	an exKMP.
\end{frame}

\section{Manacher}

\begin{frame}
	a Manacher.
\end{frame}

\section{Trie}

\begin{frame}
	
\end{frame}

\end{document}