\documentclass[table]{beamer}

\usepackage{graphicx, graphics}
\usepackage{float, array, color, ctex}
\usepackage{amsmath, amssymb}
\usepackage{multicol, multirow, makecell, tabu, dcolumn}
\usepackage{fancyhdr, lastpage}
\usepackage{listings, xcolor}
\usepackage{xeCJKfntef}
\usepackage{fontspec, xunicode, xltxtra}
\usepackage{setspace}
\usepackage{geometry}

\usepackage[UTF8, noindent]{ctexcap}
\usetheme{Szeged}
\usecolortheme{beaver}
\usefonttheme[math]{serif}

\title{莫队}
\author{张敬东}
% \institute{福州第十六中学}
\date{\today}

\AtBeginSection[]
{
	\begin{frame}
		\frametitle{目录}
		\tableofcontents[currentsection]
	\end{frame}
}

\begin{document}

\frame{\titlepage} % 生成标题页

\begin{frame} % 生成目录页
	\frametitle{目录}
	\tableofcontents
\end{frame}

\section{前言}
\begin{frame}
	今天的题单:{\color{blue}\href{https://www.luogu.com.cn/training/428639}{点我}}

	今天是我第一次上台,ppt 可能不是那么美观,如果有错误请大胆指出。

	例题不会很难,请放心食用。
\end{frame}

\section{简介}

\begin{frame}[fragile]{莫队简介}
	什么是莫队?

	莫队是由莫涛大神提出的一种\sout{暴力}区间操作算法,它框架简单、板子好写、复杂度优秀。

	然而由于莫队算法应用的毒瘤,很多莫队题难度评级都很高(蓝紫黑),
	令很多初学者望而却步。但如果你真正理解了莫队的算法原理,它用起来还是挺简单的。
	\pause

	\begin{block}{前置知识}
		\begin{itemize}
			\item 分块思想。
			\item \verb|sort| 的用法
			(包括重载运算符或 \verb|cmp| 函数,多关键字排序)。
			\item And so on.
		\end{itemize}
	\end{block}
\end{frame}

\begin{frame}{使用莫队的情境}
	若 $m=O(n)$(即 $m$、$n$ 同阶),且\textbf{$[l,r]$ 的答案能 $O(1)$ 地转换到 
	$[l-1,r],[l,r+1],[l+1,r],[l,r-1]$ 区间(即相邻区间)的答案},
	那么莫队可以在 $O(n\sqrt{n})$ 的时间复杂度内\textbf{离线}求出所有询问的答案。
	\pause

	\begin{alertblock}{注意}
		莫队是离线算法。如果题目强制在线,则不以可用莫队。
	\end{alertblock}

	\begin{block}{什么是离线、在线?}
		如果算法需要知道所有询问才能开始算法,则称此算法为离线算法。

		读入一个询问,回答一个询问的算法叫在线算法。

		强制在线就是要求你读入一个询问就立马回答。
	\end{block}
\end{frame}

\begin{frame}{莫队的基本思想}\Large
	\textbf{离线存下所有询问,借助分块按照一定的顺序处理这些询问,使得询问之间可以互相利用
	(一般情况下为了方便,只会是本次询问利用上次询问的答案),以减小时间复杂度。}
\end{frame}

\section{普通莫队}

\subsection{算法基础}

\begin{frame}{算法流程}
	\begin{enumerate}
		\item 离线存下所有询问。
		\pause

		\item 以二元组 $(bel[l_i],r_i)$ 为关键字升序对所有询问排序。
		
		其中 $i$ 表示当前询问编号,$bel[l_i]$(belong,属于)表示 $l_i$ 所在的块的编号。
		\pause

		\item 遍历每个询问,维护两个指针 $[l,r]$ 表示当前区间,$tmpans$ 表示当前答案。
		
		初始时 $l=1,r=0$(如果 $l=0$,那么我们还需要删除 $a_0$,导致一些奇怪的错误)。
		
		\textbf{$l,r$ 需区别于 $l_i,r_i$,它们一对是我们维护的指针(下标),
		一对是数据给出的询问。}
		\pause

		\item 移动区间 $[l,r]\to [l_i,r_i]$。途中维护区间 $[l,r]$ 的答案 $tmpans$。
		\pause

		\item 移动结束后,记录区间 $[l_i,r_i]$ 的答案 $ans_i$。
		
		($ans_i\gets tmpans$)。
	\end{enumerate}
\end{frame}

\begin{frame}[fragile]
	{算法代码:{\color{blue}
	\href{https://www.luogu.com.cn/problem/P3901}
	{洛谷 P3901 数列找不同}}(模板题)\footnotesize【有其他解法】}
	\tiny
	\begin{verbatim}
constexpr int N=114514;
int n,m,a[N],S;// S:块长
// [1,S] 区间的块编号为 1,[S+1,2S] 区间的块编号为 2,以此类推。
inline int bel(int x){return (x-1)/S+1;}
struct query// 询问结构体
{
    int l,r,id;// 分别为每个询问区间的左端点、右端点、询问的编号。
    friend inline bool operator < (query x,query y)
    {return (bel(x.l)==bel(y.l)?x.r<y.r:bel(x.l)<bel(y.l));}
};
query q[N];// 查询数组
bitset<N> ans;// 答案数组
// cnt[i]:i 这个数在当前区间 [l,r] 出现次数,cf:重复出现的数的数量。
// 如果 cf=0,[l,r] 中没有重复出现的数。
int cnt[N],cf=0;
// 移动区间
inline void add(int pos)// 添加 a[pos]
{
    cnt[a[pos]]++;// 将 a[pos] 的出现次数 +1。
    if(cnt[a[pos]]==2)cf++;// 如果已经出现两次,则重复了,cf++。
}
inline void del(int pos)// 删除 a[pos]
{
    cnt[a[pos]]--;// 将 a[pos] 的出现次数 -1。
    if(cnt[a[pos]]==1)cf--;// 如果当前只出现一次,则之前一定重复(出现两次),
    // 而现在不重复了,cf--。
}
\end{verbatim}
\end{frame}

\begin{frame}[fragile]
	{算法代码:{\color{blue}
	\href{https://www.luogu.com.cn/problem/P3901}
	{洛谷 P3901 数列找不同}}(模板题)\footnotesize【有其他解法】}
	\tiny
	\begin{verbatim}
void mt()
{
    S=int(ceil(pow(n,0.5)));// S=sqrt(n),根号分块
    sort(q+1,q+m+1);// 结构体排序
    for(int i=1,l=1,r=0;i<=m;i++)// 遍历每个询问
    {
        #define q q[i]// 偷懒
        while(q.l<l)add(--l);// 向左扩展 l-1
        while(r<q.r)add(++r);// 向右扩展 r+1
        while(l<q.l)del(l++);// 向右删除 l
        while(q.r<r)del(r--);// 向左删除 r
        // 注意上面四句的顺序,需要先扩展在删除。
        // 同时注意自减自加运算符是前置还是后置。
        ans[q.id]=!cf;// 记录当前答案
        #undef q
    }
}

int main()
{
    // input
    mt();// 莫队
    for(int i=1;i<=m;i++)puts(ans[i]?"Yes":"No");// 输出
    return 0;
}
\end{verbatim}
\end{frame}

\begin{frame}{算法复杂度}
	下面的讨论中 $m=O(n)$。
	
	单次移动 $l,r$ 中的一个复杂度显然 $O(1)$。
	\pause

	考虑 $l,r$ 分别移动的次数。
	\pause

	\begin{itemize}
		\item 考虑 $l$:设块 $i$ 内的询问的左端点个数为 $x_i$,则块 $i$ 中移动 $l$ 的次数
		顶多 $x_i\times\sqrt{n}$。一共 $\sqrt{n}$ 个块,移动 $l$ 的总时间复杂度为:
		$$\begin{aligned}
			&\sum_{i=1}^{\sqrt{n}} \left(x_i\times\sqrt{n}\right)\\
			=&\left(\sqrt{n}\right)\times\sum_{i=1}^{\sqrt{n}} \left(x_i\right)\\
			=&\sqrt{n}\times m\\
			=&O(n\sqrt{n})
		\end{aligned}$$
	\end{itemize}
\end{frame}
\begin{frame}{算法复杂度}
	\begin{itemize}
		\item 考虑 $r$:每块内的 $x_i$ 个 $l_j(1\le j\le x_i)$ 肯定一一对应着 $x_i$ 个 $r_j$。
		显然这 $x_i$ 个 $r_j$ 最多会使 $r$ 移动 $n$ 的长度($l_j$ 同一块时,按 $r_j$ 升序,故不降)。

		一共 $\sqrt{n}$ 个块,移动 $r$ 的总时间复杂度为:$\sqrt{n}\times n=O(n\sqrt{n})$。
		\pause

		\item 则总时间复杂度为 $O(1)\times [O(n\sqrt{n})+O(n\sqrt{n})]=O(n\sqrt{n})$。
	\end{itemize}
\end{frame}

\subsection{算法优化}

\begin{frame}[fragile]{奇偶性排序}
	刚才的复杂度分析中提出了一些极端情况:$x_i$ 个 $r_j$ 最多使 $r$ 移动 $n$ 的长度。
	而 $\sqrt{n}$ 个块都可能使 $r$ 移动 $n$ 的长度,例如下列询问
	(\textbf{已排序且块长为 $4$}):
\begin{verbatim}
1 1
2 100
3 1
4 100
\end{verbatim}
	\pause
	按原先的排序策略,$r$ 需要反复横跳,十分浪费时间。
	如果将处理顺序改为以下顺序将大大加速:
\begin{verbatim}
1 1
2 100
4 100
3 1
\end{verbatim}

\end{frame}

\begin{frame}{奇偶性排序}

	怎么修改排序策略?

	\textbf{依然以 $bel[l_i]$ 为第一关键字升序排序,
	若 $bel[l_i]$ 为奇数,以 $r_i$ 为第二关键字升序排序,
	反之若 $bel[l_i]$ 为偶数,以 $r_i$ 为第二关键字降序排序。}
	\pause

	通俗来讲:即对于属于奇数块的询问,$r$ 按从小到大排序,对于属于偶数块的排序,$r$ 从大到小排序。

	这样我们的 $r$ 指针在处理完这个奇数块的问题后,将在返回的途中处理偶数块的问题,
	再向 $n$ 移动处理下一个奇数块的问题,优化了 $r$ 指针的移动次数,
	一般情况下,这种优化能让程序快 $30\%$ 左右。——OI-Wiki。
	\pause

	实测:\href{https://www.luogu.com.cn/record/137463623}{\color{blue}$810$ ms} $\to$
	\href{https://www.luogu.com.cn/record/137467226}{\color{blue}$622$ ms}。快约 $23.2\%$。

\end{frame}

\begin{frame}[fragile]{奇偶性排序:代码}
\begin{verbatim}
struct query
{
    int l,r,id;
    friend inline bool operator < (query x,query y)
    {
        if(bel(x.l)!=bel(y.l))return bel(x.l)<bel(y.l);
        if(bel(x.l)&1)return x.r<y.r;
        else return x.r>y.r;
    }
};
\end{verbatim}
\end{frame}

\begin{frame}[fragile]{常数级展开}
	发现 \verb|add()|,\verb|del()| 两个函数可以压行并展开到 \verb|mt()| 中。
	
	这看似鸡肋的优化
	实测 \href{https://www.luogu.com.cn/record/137468354}{\color{blue}$572$ ms}——又优化了 $50$ ms。
	\pause

	代码:\scriptsize
\begin{verbatim}
void mt()
{
    S=int(ceil(pow(n,0.5)));
    sort(q+1,q+m+1);
    for(int i=1,l=1,r=0;i<=m;i++)
    {
        #define q q[i]
        // 压行并展开:
        while(q.l<l)cf+=(++cnt[a[--l]]==2);// 与 add(--l) 等价
        while(r<q.r)cf+=(++cnt[a[++r]]==2);// 与 add(++r) 等价
        while(l<q.l)cf-=(--cnt[a[l++]]==1);// 与 del(l++) 等价
        while(q.r<r)cf-=(--cnt[a[r--]]==1);// 与 del(r--) 等价
        ans[q.id]=!cf;
        #undef q
    }
}
\end{verbatim}
\end{frame}

\begin{frame}[fragile]{玄学剪枝}
	我考虑到有时候可能转移大半天还不如暴力重新算,所以想出了一个玄学剪枝:
	\begin{verbatim}
// 如果转移代价大于重新算的代价
if(abs(q.l-l)+abs(q.r-r)>(r-l+1)+(q.r-q.l+1))
{
    while(l<=r)cf-=(--cnt[a[r--]]==1);// 清除
    r=(l=q.l)-1;// 直接跳转
}
\end{verbatim}
	(这段语句加在 \verb|#define q q[i]| 后面。)

	没想到只优化了 $1$ ms(悲。也许是每次都判断的代价太大,抵消了直接跳转的优化。
\end{frame}

\subsection{例题}

\begin{frame}
{例题:{\color{blue}\href{https://www.luogu.com.cn/problem/SP3267}{DQUERY - D-query}}}
	简要题意:给出一个长度为 $n$ 的数列 $a$,$m$ 个询问,每次询问给出数对 $l,r$ 表示
	查询区间 $[l,r]$ 中有多少不同的数。

	数据范围:$n\le 3\times10^5,m\le2\times10^5,a_i\le10^6$。
	\pause

	\textbf{板子题,难度在于如何 $O(1)$ 转移答案。}
	\pause

	考虑用数组 $cnt_i$ 表示 $[l,r]$ 中 $i$ 出现了几次,
	变量 $bt$ 表示 $[l,r]$ 中有多少不同的数。
	
	要添加 $a_{pos}$,那么 $cnt[a_{pos}]\gets cnt[a_{pos}]+1$。

	此时若 $cnt[a_{pos}]=1$,即多了一个不同的数,那么 $bt\gets bt+1$。

	同理删除 $a_{pos}$ 时 $cnt[a_{pos}]\gets cnt[a_{pos}]-1$,
	若 $cnt[a_{pos}]=0$(少了一个数),$bt\gets bt-1$。

	其他的正常地跑莫队即可。\textbf{但此题似乎卡常。}
\end{frame}

\begin{frame}
{例题:{\color{blue}\href{https://www.luogu.com.cn/problem/P2709}{P2709 小B的询问}}}
	简要题意:给出一个长度为 $n$ 的数列 $a$(值域 $[1,k]$),$m$ 个询问,每次询问给出数对 $l,r$ 表示
	查询:  
	$$\sum\limits_{i=1}^k c_i^2$$
	其中 $c_i$ 表示数字 $i$ 在 $[l,r]$ 中的出现次数。  

	数据范围:$1\le n,m,k \le 5\times 10^4$。
	\pause

	\textbf{难度依然在于如何 $O(1)$ 转移答案。}
	\pause

	$c$ 数组很好维护,但答案(设它为 $s$)就不那么好维护了。

	由于每次添加或删除数时只会改变 $c_{a[pos]}$,而且只会 $\pm1$。所以:
\end{frame}

\begin{frame}
{例题:{\color{blue}\href{https://www.luogu.com.cn/problem/P2709}{P2709 小B的询问}}}
	由
	$$
	s=\sum\limits_{i=1}^k c_i^2=c_1^2+\cdots+c_{a[pos]}^2+\cdots+c_k^2
	$$
	可得
	$$
	\begin{aligned}
		s'&=c_1^2+\cdots+(c_{a[pos]}+1)^2+\cdots+c_k^2\\
		&=c_1^2+\cdots+c_{a[pos]}^2\pm2\times c_{a[pos]}+1+\cdots+c_k^2\\
		&=s\pm2\times c_{a[pos]}+1
	\end{aligned}
	$$
	转移时修改即可。
\end{frame}

\section{带修莫队}

\subsection{简介}
\begin{frame}{如何实现带修莫队?}
	发现普通莫队不支持修改,那么如何使它支持修改操作呢?

	考虑存询问时加一个变量 $t_i$ 表示\textbf{进行此询问时前面修改了几次}。
	同时存下每一个修改操作(无需排序)。

	再新增一个指针 $t$ 表示当前区间所在的时间位置。那么移动方向就从 $4$ 个变为 $6$ 个:
	$[l-1,r,t],[l,r+1,t],[l+1,r,t],[l,r-1,t],[l,r,t-1],[l,r,t+1]$。新增的两个为时间轴上的移动。
\end{frame}

\subsection{例题}
\begin{frame}[fragile]
{例题:{\color{blue}\href{https://www.luogu.com.cn/problem/P1903}{P1903 [国家集训队] 数颜色 / 维护队列}}}
	简要题意:给出一个长度为 $n$ 的数列,$m$ 个操作,要求支持两种操作:查询区间有多少不同的数、单点修改。

	数据范围:$n,m\le 1.33333\times 10^5,a_i\le 10^6$。
	\pause

	板子题,直接上代码:\tiny
\begin{verbatim}
constexpr int N=214514,A=1145141;// A:a 的值域
int n,m,S,qm,a[N];// qm:询问的个数
inline int bel(int x){return (x-1)/S+1;}// 分块
struct query
{
    int l,r,t,id;// 额外记录时间
    friend inline bool operator < (query x,query y)
    {//        若 l 所在块不同    按 l 的块的编号 否则 若 r 所在块不同 按 r 的块的编号 否则按时间排
        return (bel(x.l)^bel(y.l)?bel(x.l)<bel(y.l):(bel(x.r)^bel(y.r)?bel(x.r)<bel(y.r):x.t<y.t));
    }
};query q[N];
struct modify// 新增:修改操作
{int p,v;};modify mo[N];
\end{verbatim}
\end{frame}
\begin{frame}[fragile]
{例题:{\color{blue}\href{https://www.luogu.com.cn/problem/P1903}{P1903 [国家集训队] 数颜色 / 维护队列}}}
\tiny
\begin{verbatim}
int cnt[A],bt,ans[N];// 类似于普通莫队
void mt()
{
    S=int(ceil(pow(n,0.66)));// 这里块长需要调整,具体可以可以看
    // https://oi-wiki.org/misc/modifiable-mo-algo/ 中的证明
    sort(q+1,q+qm+1);
    for(int i=1,l=1,r=0,t=0;i<=qm;i++)
    {
        #define q q[i]
        #define p mo[t].p
        #define v mo[t].v
        while(q.l<l)bt+=(!(cnt[a[--l]]++));// 类似
        while(r<q.r)bt+=(!(cnt[a[++r]]++));
        while(l<q.l)bt-=(!(--cnt[a[l++]]));
        while(q.r<r)bt-=(!(--cnt[a[r--]]));
        // 可怕的压行:【需要当场解释】
        while(t<q.t){t++;if(l<=p&&p<=r)bt-=(!(--cnt[a[p]])-!(cnt[v]++));swap(a[p],v);}
        while(q.t<t){if(l<=p&&p<=r)bt-=(!(--cnt[a[p]])-!(cnt[v]++));swap(a[p],v);t--;}
        ans[q.id]=bt;
        #undef q
        #undef p
        #undef v
    }
}
\end{verbatim}
\end{frame}

\end{document}