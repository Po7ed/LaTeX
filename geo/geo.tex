\documentclass[b5paper,10pt]{book}

\usepackage{graphicx, graphics}
\usepackage{float, array, color, ctex}
\usepackage{amsmath, amssymb}
\usepackage{multicol, multirow, makecell, tabu, dcolumn}
\usepackage{fancyhdr, lastpage}
\usepackage{listings, xcolor}
\usepackage{xeCJKfntef}
\usepackage{fontspec, xunicode, xltxtra}
\usepackage{setspace}
\usepackage{geometry}
\usepackage{multirow}
\usepackage{siunitx}

\usepackage[UTF8, noindent]{ctexcap}

\title{Geography}
\author{张敬东}
\date{\today}
% \fontfamily{songti}

\begin{document}
\pagenumbering{roman}
\maketitle
\newpage\newpage

\tableofcontents
\newpage 
\pagenumbering{arabic}
\setcounter{page}{1}

\part{七上}
\newpage
tmp
\part{七下}
\newpage
tmp
\part{八下}
\newpage

\chapter{从世界看中国}
\section{疆域}
\subsection{优越的地理位置}

\begin{enumerate}
	\item 半球:东、北半球。
	\item 纬度:大部分北温带,南部热带,无寒带。
	\item 海陆:亚欧大陆东部,太平洋西岸,海陆兼备。
\end{enumerate}

\subsection{海陆兼备的大国}

领土四至:
\begin{itemize}
	\item 南:海南省南沙群岛曾母暗沙。
	\item 北:黑龙江省漠河市北端黑龙江(主航道中心线)。
	\item 东:黑龙江省黑龙江与乌苏里江(主航道中心线)交汇处。
	\item 西:新疆维吾尔自治区帕米尔高原。
\end{itemize}

南北气候差异显著,东西时间差异大。

领土面积 $960$ 万平方千米。

与 $14$ 个国家相邻。

海域:
\begin{itemize}
	\item 濒临的海洋(从北到南):渤海、黄海、东海、南海。
	\item 内海:渤海、琼州海峡。
\end{itemize}

\subsection{行政区划}

实行:省、县、乡三级区划。
我国有 34 个省级行政区。

一些特殊的省级行政区:
\begin{itemize}
	\item 北回归线上:云南省、广西壮族自治区、广东省、台湾省。
	\item 东西跨度最大:内蒙古自治区。
	\item 南北跨度最大:云南省。
	\item 面积最大:新疆维吾尔自治区。
	\item 少数民族最多:云南省。
\end{itemize}

\section{人口}
\subsection{世界上人口最多的国家}

人口众多:
\begin{itemize}
	\item 利:劳动力、消费市场。
	\item 弊:交通、资源、环境、社会经济:压力。
\end{itemize}

\subsection{人口东多西少}
胡焕庸线(黑龙江黑河—云南腾冲):东部人口稠密,北部人口稀疏。

\section{民族}
\subsection{中华民族大家庭}
汉族:$91\%$,壮族次之。

传统节日:
\begin{itemize}
	\item 壮族:“三月三”歌节。
	\item 傣族:泼水节。
	\item 苗族:苗年。
	\item 蒙古族:那达慕节。
	\item 藏族:雪顿节。
\end{itemize}

非遗:
\begin{itemize}
	\item 新疆维吾尔木卡姆艺术。
	\item 侗族大歌。
	\item 藏族史诗《格萨尔王传》。
	\item 蒙古族呼麦。
\end{itemize}

\subsection{民族分布特点}
\begin{itemize}
	\item 汉族:遍布全国,主要:中、东部。
	\item 少数民族:西南、西北、东北。
\end{itemize}

特点:大杂居,小聚居,交错居住。

\chapter{中国的自然环境}
\section{地形和地势}
\subsection{地形类型多样,山区面积广大}

地形特点 $\uparrow$。

山区:山地、丘陵、崎岖的高原。
\begin{itemize}
	\item 利:发展林业、牧业、旅游业、采矿业。
	\item 弊:地形崎岖、交通不便、耕地不足。
\end{itemize}

\subsection{地势东高西低,呈阶梯状分布}
地势特点 $\uparrow$。

利:
\begin{itemize}
	\item 便于湿润气流深入内陆,形成降水(农业)。
	\item 河流众多,便利东西交通(交通)。
	\item 阶梯边缘落差大,水能资源丰富(能源)。
\end{itemize}

\section{气候}
\subsection{冬季南北温差大,夏季普遍高温}

冬季 $\qty{0}{\degreeCelsius}$ 等温线大致为:秦岭—淮河。极差约 $\qty{50}{\degreeCelsius}$。

夏季只有青藏高原等少数地区气温低。

冬季最冷:黑龙江省漠河市的北极村。
夏季最热:新疆维吾尔自治区的吐鲁番。

温度带与熟制($5+1$,用 $(x,y)$ 表示 $x$ 年 $y$ 熟):
\begin{itemize}
	\item 寒温带:$(1,1)$。
	\item 中温带:$(1,1)$。
	\item 暖温带:$(2,3)|(1,2)$。
	\item 亚热带:$(1,2)|(1,3)$。
	\item 热带:$(1,3)$。
	\item 青藏高原区:$(1,1)$。
\end{itemize}

\subsection{东西干湿差异显著}

总趋势:从东南沿海向西北内陆递减。

降水最多:台湾省东北部的火烧寮。

降水最少:新疆维吾尔自治区的吐鲁番盆地中的托克逊。

雨季:$4\sim10$ 月。

\begin{itemize}
	\item 南方雨季开始早、结束晚、雨季长、雨量大。
	\item 北方雨季开始晚、结束早、雨季长、雨量小。
\end{itemize}

\begin{center}
	\begin{tabular}{|c|c|c|c|}
		\hline
		\textbf{干湿区} & \textbf{降水量} & \textbf{植被} & \textbf{农业}\\
		\hline
		湿润区 & $>\qty{800}{\mm}$ & 森林 & 水稻\\
		\hline
		半湿润区 & $\qty{400}{\mm}\sim\qty{800}{\mm}$ & 森林、草原过渡 & 小麦\\
		\hline
		半干旱区 & $\qty{200}{\mm}\sim\qty{400}{\mm}$ & 草原 & 放牧业\\
		\hline
		干旱区 & $<\qty{200}{\mm}$ & 沙漠、戈壁 & 放牧业、灌溉农业\\
		\hline
	\end{tabular}
\end{center}

\part{八下}
\newpage
tmp
\end{document}