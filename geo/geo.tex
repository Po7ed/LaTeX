\documentclass[12pt,a4paper]{article}

\usepackage{graphicx, graphics}
\usepackage{float, array, color, ctex}
\usepackage{amsmath, amssymb}
\usepackage{multicol, multirow, makecell, tabu, dcolumn}
\usepackage{fancyhdr, lastpage}
\usepackage{listings, xcolor}
\usepackage{xeCJKfntef}
\usepackage{fontspec, xunicode, xltxtra}
\usepackage{setspace}
\usepackage{geometry}
\usepackage{multirow}

\usepackage[UTF8, noindent]{ctexcap}

\title{Geography}
\author{张敬东}
\date{\today}
% \fontfamily{songti}

\begin{document}
\pagenumbering{roman}
\maketitle
\newpage\newpage

\tableofcontents
\newpage

\pagenumbering{arabic}
\setcounter{page}{1}

\section{八下}
\subsection{从世界看中国}
\subsubsection{疆域}
\paragraph*{优越的地理位置}

\begin{enumerate}
	\item 半球:东、北半球。
	\item 纬度:大部分北温带,南部热带,无寒带。
	\item 海陆:亚欧大陆东部,太平洋西岸,海陆兼备。
\end{enumerate}

领土四至:
\begin{itemize}
	\item 南:海南省南沙群岛曾母暗沙。
	\item 北:黑龙江省漠河市北端黑龙江(主航道中心线)。
	\item 东:黑龙江省黑龙江与乌苏里江(主航道中心线)交汇处。
	\item 西:新疆维吾尔自治区帕米尔高原。
\end{itemize}

南北气候差异显著,东西时间差异大。

领土面积 960 万平方千米。

与 14 个国家相邻。

海域:
\begin{itemize}
	\item 濒临的海洋(从北到南):渤海、黄海、东海、南海。
	\item 内海:渤海、琼州海峡。
\end{itemize}

\paragraph*{行政区划}

实行:省、县、乡三级区划。
我国有 34 个省级行政区。

\end{document}